\begin{document}

  Tipkovnice predstavljajo pomemben del računalniške opreme, velikokrat pa so se razvijale ob izumu novih tehnologij.
  Prva znana postavitev tipkovnice je bila alfabetična, ki so jo kasneje imenovali kar ABCDE postavitev. \cite{keyboard_history}
  Druga bolj pomembna postavitev je bila tako imenovana QWERTY postavitev, ki jo je iznašel Christopher Sholes leta 1874 in je bila del pisalnega stroja Remington No. 1.
  Leta 1932 sta Dr. August Dvorak in Dr. William Dealey iznašla in patentirala postavitev Dvorak, ki naj bi nadomestila zastarelo QWERTY postavitev.
  Kasneje so nastale tudi druge postavitve, ki so omenjene bolj podrobno v naslednjem poglavju. \\

  Že proti koncu devetnajstega stoletja so postajali popularni pisalni stroji.
  Le-ti so mehanične naprave, opremljenje s tipkami.
  Pritisk na tipko povzroči, da se na list papirja preko traku s črnilom spusti ročica s privzdignjeno črko,
  ki na papirju pusti odtis črnila.
  Ob vsakem pritisku črke se papir na nosilu prestavi za velikost ene črke v levo, kar omogoči, da je na naslednjem mestu odtisnjena naslednja črka.
  Vsaka izmed tipk odtisne drugačno črko, pisalni stroj pa ima tudi posebne tipke.
  Primeri teh so \emph{preslednica}, ki prestavi papir za velikost ene črke naprej,
  \emph{dvigalka} (ang. \emph{Shift}), ki začasno omogoči odtis velikih tiskanih črk, tako da začasno zadrži boben v zgornjem položaju za zapis alternativega znaka, večinoma kar velike tiskane črke.
  Kasneje so pisalni stroji omogočali tudi trajni privzdig bobna v zgornji položaj za zapis večih zaporednih alternativnih znakov oz. velikih tiskanih črk (\emph{zaklepalka} - ang. \emph{Caps Lock}).
  Dve drugi tipki, ki sta bili prisotni na kasnejših modelih pisalnih strojev, sta \emph{tabulator}, ki je omogočal preskok na naslednjega izmed osmih prednastavljenih mest, in
  \emph{vračalka} (ang. \emph{Backspace}), ki je premaknila papir za eno mesto nazaj, kar je omogočilo popravke zmotno napisanih črk.
  Prvi pisalni stroji so vsebovali ABCDE postavitev tipk, saj so bili ljudje najbolj navajeni abecednega vrstnega reda.
  Z izumom hitrostnega tipkanja so se zaradi ponesrečene postavitve tipke pogosto zatikale, kar je otežilo pisanje in uničevalo papir.
  Ta problem je leta 1874 rešil Christopher Sholes s svojo QWERTY tipkovnico, ko je najbolj pogoste pare tipk (v angleškem jeziku) postavil na nasprotne konce,
  da bi tako preprečil zatikanje sosednjih tipk.

\end{document}