\begin{document}

  Računalniške tipkovnice predstavljajo del našega vsakdana.
  Današnje tipkovnice imajo korenine v prvih računalnikih,
  kjer postavitve tipk niso bile prilagojene za ergonomično in učinkovito uporabo.
  Dandanes ko nam je življenje brez računalnikov nepredstavljivo in
  jih uporabljamo dnevno tudi po več ur, je še toliko bolj pomembno, da nam ponujajo ergonomično podporo
  in zmanjšujejo možnosti zdravstvenih težav, ki se pojavijo kot posledica njihove uporabe.
  Kljub številčnim raziskavam na tem področju,
  so danes v uporabi še vedno večinoma zastarele in neoptimizirane postavitve tipk tipkovnice.

  Namen moje seminarske naloge je proučiti rezultate teh raziskav in uporabiti njihove ugotovitve pri oblikovanju
  optimizirane postavitve tipk za slovenski jezik.

\end{document}