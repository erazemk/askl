\begin{document}

    \subsection{Načela odločanja}\label{subsec:načela-odločanja}

    Svojo postavitev tipkovnice sem naredil na podlagi načel, ki jih je postavil August Dvorak v svoji raziskavi
    in na podlagi predlogov dveh izdelovalcev postavitev tipkovnic, Martina Krzywinskija,
    programerja Carpalx programa in Rolanda Kaufmanna, ki je ustvaril Programersko Dvorak postavitev.
    Načela so sledeča \cite{roland_mail}:

    \begin{itemize}
        \item alteracija med rokama mora biti čim bolj pogosta,
        \item desna roka je pri večini ljudi močnejša, zato naj opravi več dela,
        \item večina tipkanja naj se opravi na srednji (domači) vrstici,
        \item najpogostejše uporabljene črke in kombinacije dveh črk naj bodo na desni strani srednje vrstice,
        \item najredkejše uporabljene črke naj bodo na spodnji vrstici.
    \end{itemize}

    \subsection{Postopek izdelave postavitve}\label{subsec:postopek-izdelave-postavitve}

    Za ugotavljanje najbolše postavitve sem se odločil uporabiti program Carpalx,
    saj kljub razmeroma preprosti uporabi omogoča spreminjanje nastavitev,
    s pomočjo tako imenovanih uteži (ang. \emph{weights}) in "kazni" (ang. \emph{penalties}) za pritisk tipk z različnimi prsti.

    Program je napisan v programskem jeziku Perl in je na voljo za več operacijskih sistemov.
    Optimalno postavitev tipkovnice program najde tako, da pregleda vnešeno postavitev tipk,
    prebere korpus iz tekstovne datoteke (sklop besedila) in nato pri računanju doda uteži in kazni \cite{typing_effort}.
    Program izračuna težavnost za vnešeno postavitev, nato zamenja mesti dveh črk in ponovi izračun.
    Po določenem številu zamenjav program izpiše postavitev z najmanjšo težavnostjo,
    ki je torej optimizirana glede na podano postavitev, korpus izbranega besedila in določene dodatne nastavitve.

    \subsubsection{Težavnostni model}

    Program Carpalx uporablja tako imenovan težavnostni model oz. model vloženega napora (ang. \emph{typing effort model}), ki izračuna napor pri uporabi posamezne postavitve.
    Glede na podan korpus ima vsaka postavitev izračunano določeno težavnost oz. vloženi napor, nato pa program z večimi poskusi najde postavitev z najmanjšo težavnostjo in jo izbere kot optimalno.
    Program omogoča, da uporabnik določi parametre težavnosti in s tem izračuna katera postavitev je subjektivno najboljša.
    Ta način izračuna je zelo uporaben, saj vsak uporabnik tipka na drugačen način, program pa pomaga tudi uporabnikom z določenimi nezmožnostmi,
    kot so poškodovani ali manjkajoči prsti, bolečine v rokah ob določenih gibih itd. \\

    Težavnostni model temelji na branju trojic (ang. \emph{triads}), tj. zaporedju treh črk iz besedila.
    Nato model preveri razdaljo med tipkami, ki jo morajo opraviti prsti, prišteje vnaprej določene kazni in uteži in pregleda obliko poti,
    ki jo opravijo prsti (ali so gibi roke enostavno izvedljivi ali zapleteni).
    Težavnost je nato izračunana za posamezno tipko in za skupino zaporednih tipk.
    Pri posamezni tipki je težavnost izračunana predvsem glede na razdaljo med zaporednima pritiskoma, k temu pa so prištete uteži in kazni.
    Pri skupini tipk na težavnost vpliva oblika poti, ki pove kako zapletene gibe mora narediti roka, da doseže zaporedje teh tipk.

    \subsubsection{Trojice}

    Program omogoča nastavitev prekrivanja trojic, koncept, ki je prikazan na primeru spodaj.
    Besedo "abeceda" lahko razstavimo na naslednje trojice

    \begin{verbatim}
        abeceda

        abe-
        -bec-
         -ece-
          -ced-
           -eda
    \end{verbatim}

    Na tem primeru lahko vidimo, da se prekrivata po dve črki, kar je privzeta nastavitev programa.
    Težavnost je nato izračunana kot povprečje seštevka težavnosti za posamične trojice: \\

    $E = \frac{1}{N} \displaystyle\sum_{i} e_{i} = \frac{1}{N} \displaystyle\sum_{i, i \neq j \implies e_{i} \neq e_{j}} n_{i} e_{i}$ \\

    kjer je $N$ število vseh trojic, $t_{i}$ je težavnost tipkanja trojice $i$.
    Število unikatnih trojic je omejeno, zato je najbolj učinkovito narediti indeks vseh trojic in nato sešteti število ponovitev ($n_{i}$).
    Težavnost vsake posamične trojice se izračuna po formuli: \\

    $e_{i} = k_{b} b_{i} + k_{p} p_{i} + k_{s} s_{i}$ \\

    kjer so $b_{i}$, $p_{i}$ in $s_{i}$ izračuni osnove, kazni in oblike poti za trojico $i$.
    Osnovo in kazen izračunamo po formulah: \\

    $b_{i} = k_{1} b_{i1} (1 + k_{2} b_{i2} (1 + k_{3} b_{i3}))$

    $p_{i} = k_{1} p_{i1} (1 + k_{2} p_{i2} (1 + k_{3} p_{i3}))$

    \subsection{Analiza postavitve}\label{subsec:analiza-postavitve}

    Analiza temelji na izračunu pogostosti uporabe vrstic, prstov, rok in preskokov vrstic:

    \begin{itemize}
        \item vrstica - pogostost tipkanja na zgornji ($row_{t}$), srednji ($row_{h}$) in spodnji ($row_{b}$) vrstici,
        \item roka - pogostost tipkanja z levo ($h_{L}$) in desno ($h_{R}$) roko,
        \item prsti - pogostost tipkanja z določenim prstom ($f_{0}$ - $f_{9}$),
        \item menjava rok - definirana z $h_{L} - h_{R}$, kjer imajo optimizirane postavitve nizko število,
        \item preskok vrstic - število preskokov znotraj trojice, npr. trojica "edc" ima preskok dveh vrstic, saj se s prstom premaknemo iz najvišje vrstice eno vrstico nižje, nato pa še na najnižjo vrstico.
    \end{itemize}

\newpage
    \subsubsection{Šumniki}\label{sumniki}

    Ker program ne podpira šumnikov, imamo možnost, da jih pri analizi zamenjamo z drugimi znaki, ki jih najdemo na ASCII tabeli.
    V izbranem korpusu so bili zamenjani naslednji šumniki:

    \begin{verbatim}
        č => \
        š => @
        ž => |
        đ => [
        ć => ]
    \end{verbatim}

    Čeprav črki đ in ć nista del slovenske abecede sem ju vključil v analizo, saj sta vseeno prisotni na slovenski tipkovnici.

    \subsubsection{Računanje težavnosti obstoječe postavitve}

    Da bi vedeli, če je naša postavitev res optimizirana, jo moramo primerjati z že obstoječo postavitvijo.
    Za moj namen sem izbral slovensko QWERTZ postavitev.
    Program Carpalx prebere postavitev izbrane tipkovnice iz svoje konfiguracijske datoteke, ki se nahaja v \emph{etc/keyboards} podmapi projekta.

    Postavitev sem v datoteki \emph{etc/keyboards/slovenian.conf} definiral kot:

    \begin{verbatim}
        <keyboard>

        <row 1>
        keys    = `~ 1! 2" 3\# 4$ 5% 6& 7/ 8( 9) 0= '? +*
        fingers = 0  0  1  2   3  3  6  6  7  8  9  9  9
        </row>

        <row 2>
        keys    = qQ wW eE rR tT zZ uU iI oO pP @! [!
        fingers = 0  0  1  2  3  3  6  6  7  8  9  9
        </row>

        <row 3>
        keys    = aA sS dD fF gG hH jJ kK lL \! ]! |!
        fingers = 0  0  1  2  3  3  6  6  7  8  9  9
        </row>

        <row 4>
        keys    = <> yY xX cC vV bB nN mM ,; .: -_
        fingers = 0  0  1  2  3  3  6  6  7  8  9
        </row>

        </keyboard>
    \end{verbatim}

    \newpage
    V datoteki lahko opazimo, da se znak "!" (klicaj) večkrat ponovi.
    Ker so male in velike tiskane črke definirane na isti tipki, jih v datoteki ni potrebno definirati dvakrat, a sem to storil zaradi boljše preglednosti.
    Pri šumnikih, ki smo jih zamenjali z drugimi znaki, bi zapis obeh črk zahteval dvakrat toliko znakov, česar si zaradi omejenega števila znakov, ki jih program prepozna, ne moremo privoščiti.
    Zato te znake nadomesti klicaj, ki ne spremeni postavitve ostalih tipk, saj ga program v tem primeru prepozna kot veliko tiskano črko.
    Vsaka tipka ima dve stanji, ki sta opisani v vrstici imenovani \emph{keys} in se v datoteki stikata (npr. \textbf{1!}).
    V tem primeru to pomeni, da bo ob pritisku te tipke izpisana števka \emph{1}, če pa ob tipki pritisnemo še dvigalko (ang. Shift),
    pa bo izpisan znak \emph{!}.

    Števke v vrstici \emph{fingers} opišejo kateri prst bo pritisnil določeno tipko.
    Prsti na roki so označeni s števkami od 0 do 9, pri čemer števka 0 predstavlja levi mezinec, števka 4 levi palec, števka 5 desni palec in števka 9 desni mezinec.
    Razporeditev prstov po tipkah sledi priporočilom metode imenovane \emph{slepo tipkanje}, kjer je namen zapomniti si postavitev tipk na tipkovnici in
    tipkati, ne da vmes gledamo na tipkovnico.

    \begin{figure}[h]
    \centering
    \includegraphics[width=0.8\linewidth]{resources/layout-slovenian-fingers.png}
    \caption{Razporeditev prstov pri slovenski postavitvi}
    \end{figure}

    Na zgornji sliki barve označujejo, kateri prsti pritisnejo na katere tipke.
    Vsaka barva ustreza enemu prstu roke, začenši z levim mezincem na svetlo modri barvi.

    Ko zaženemo program le-ta izračuna težavnost za vnešeno postavitev in nam jo izpiše.
    Spodaj je prikazan okrajšan izpis težavnosti zdajšnje slovenske QWERTZ postavitve.

    \begin{verbatim}
        Keyboard effort
        -----------------------------
        all                     2.910
    \end{verbatim}

    Ta izpis nam pove, da je težavnost postavitve, kot jo izračuna program enaka 2.910.
    Po optimizaciji postavitve bomo to vrednost primerjali z novo, optimizirano vrednostjo
    in tako izračunali, koliko boljša je nova postavitev.

    \subsection{Rezultati optimizacije}\label{subsec:rezultati-optimizacije}

    Po optimizaciji postavitve je program izpisal več postavitev, izbranih izmed 10.000 kombinacij.
    Prvi izpis je tekstovna predstavitev vnešene postavitve (slovensko QWERTZ postavitev) in izgleda takole:

    \begin{verbatim}
        ------------------------------------------------------------
        ` 1 2 3 4 5 6 7 8 9 0 ' +     ~ ! " # $ % & / ( ) = ? *
        q w e r t z u i o p @ [       Q W E R T Z U I O P ! !
        x s d f g h j k l \ ] |       X S D F G H J K L ! ! !
        < y a c v b n m , . -         > Y A C V B N M ; : _
        ------------------------------------------------------------
    \end{verbatim}

    V mislih je potrebno imeti, da so šumniki zaradi delovanja programa zamenjani z drugimi znaki, kot je navedeno v poglavju \ref{sumniki}.
    Program je nato izpisal več drugih postavitev, a nas zanima predvsem zadnji izpis, ki prikaže najboljšo postavitev.
    Le-ta je v okrajšani obliki prikazan spodaj. Število 9817 za besedo \emph{iter} (slo. ponovitev) nam pove,
    kateri poskus oprimizacije je pridobil to težavnost.
    V tem primeru smo pridobili optimalno postavitev po 9817 od 10000 poskusov.
    Števili prikazani za besedo \emph{effort} (slov. težavnost) prikazujeta stanji pred in po zadnjem poskusu optimizacije,
    tj. kolikšna je bila težavnost pred tem poskusom in kakšna je nova težavnost po tem poskusu.

    \begin{verbatim}
        ------------------------------------------------------------
        iter   9817 effort 1.878665 -> 1.878233
        ------------------------------------------------------------
        ` 1 2 3 4 5 6 7 8 9 0 ' +     ~ ! " # $ % & / ( ) = ? *
        h b p r d < c u k v @ [       H B P R D > C U K V ! !
        n s i l j t a o e \ ] |       N S I L J T A O E ! ! !
        w f q z x g y m , . -         W F Q Z X G Y M ; : _
        ------------------------------------------------------------
    \end{verbatim}

        Če postavimo tipke na tipkovnico bi optimizirana postavitev izgledala takole:

        \begin{figure}[h]
        \centering
        \includegraphics[width=0.8\linewidth]{resources/layout-slovenian-optimised.png}
        \caption{Optimizirana postavitev}
        \end{figure}

        Znaki v spodnjem delu tipke prikazujejo le-to tipko v osnovnem stanju, brez pritisnjene dvigalke (ang. Shift),
        znaki v zgornjem delu pa le-to tipko skupaj s pritisnjeno dvigalko.
        Tipke, ki vsebujejo le veliko tiskano črko, so prikazane le v drugem stanju, ne pa tudi v osnovnem stanju.

    \subsection{Primerjava s slovensko QWERTZ postavitvijo}\label{subsec:primerjava-s-slovensko-qwertz-postavitvijo}

    Vidimo lahko, da je nova postavitev po predstavljenem postopku ocenjen kot 35 \% boljša od standardne slovenske QWERTZ postavitve.
    Prav tako opazimo, da je postavitev števk ostala enaka, postavitev črk pa se je popolnoma spremenila.

    Mesta števk in večine znakov so pri obeh postavitvah enaka,
    saj se v analiziranem korpusu ne pojavijo pogosto (Glej Možnosti za nadaljni razvoj in optimizacijo, poglavje \ref{subsec:večji-in-kvalitetnejši-korpus}).
    Zanimivo je, da sta se simbola < in > premaknila na mesto, kjer je prej stala črka Z\@.

\end{document}