\begin{document}

  Spodaj so opisane možnosti za nadaljnji razvoj in optimizacijo postavitve tipkovnice.

  \subsection{Večji in kvalitetnejši korpus}\label{subsec:večji-in-kvalitetnejši-korpus}

  Za korpus sem uporabil Cankarjevo dramsko delo Hlapci, z 58.336 vrsticami in 1.841.455 besedami.
  Šumniki so bili zamenjani z znaki, navedenimi v poglavju~\ref{sumniki}.
  Program bi lahko z večjim korpusom verjetno našel boljšo postavitev.
  Prav tako bi lahko z modernejšim besedilom izboljšal analizo,
  saj besedilo iz dela ne ustreza nujno današnjim standardom pisanja in stopnji razvoja jezika.

  \subsection{Drugačne uteži in kazni}\label{subsec:drugačne-uteži-in-kazni}

  Ker program omogoča nastavljanje uteži in kazni, bi bila potrebna raziskava, ki bi pokazala optimalne vrednosti teh nastavitev.
  Pri svojem delu teh uteži in kazni nisem spreminjal, kar je lahko tudi razlog za slabšo optimizacijo.

  \subsection{Analiza tretjega stanja tipk}\label{subsec:analiza-tretjega-stanja-tipk}

  Pomen besede "stanje tipk" je opisan v poglavju~\ref{subsec:stanje-tipk-tipkovnice}.
  Program Carpalx zna analizirati le prvi dve stanji tipk tipkovnice, kar pomeni,
  da program ne optimizira postavitve tipkovnicetudi za učinkovitost pri tretjem stanju tipk.
  Prireditev programa za analizo tudi tretjega stanja tipk bi pripomogla k boljši optimizaciji postavitve.

\end{document}