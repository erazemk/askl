\begin{document}

    Začetke računalniških tipkovnic lahko najdemo že ob koncu sedemnajstega stoletja,
    kot sestavni del teleprinterjev, ki so bili uporabljeni za prenašanje cen delnic na poseben papir.
    Leta 1868 je Christopher Latham Sholes patentiral prvi moderni pisalni stroj,
    kmalu zatem pa jih je podjetje Remington Company začelo množično proizvajati.
    Sholes in njegov sodelavec James Densmore sta leta 1878 patentirala zdaj ikonično
    \emph{QWERTY postavitev tipk}.
    Konkretnega razloga za izdelavo te postavitve ne poznamo, a najbolj verjetna razlaga je,
    da je Sholes videl, da so se ob hitrem tipkanju na pisalnem stroju tipke pogosto zataknile,
    zato je najpogostejše kombinacije črk postavil na nasprotne konce tipkovnice.
    Ta, sicer zastarela, postavitev se še danes uporablja v veliki večini angleških tipkovnic, iz nje pa so nastale
    tudi druge različice, kot sta na primer slovenski QWERTZ in francoski AZERTY\@. \\

    Na področju postavitve tipk tipkovnice je bilo narejenih že več raziskav,
    zanimivejšo izmed njih pa je opravil dr. August Dvorak.
    V večletni raziskavi je preučeval angleški jezik in zasnoval postavitev tipkovnice,
    ki je bila popolnoma prilagojena le-temu.
    Kljub velikemu napredku, ki ga predstavlja ta raziskava za postavitve tipk tipkovnic,
    pa dandanes postavitev Dvorak, imenovano po dr. Dvoraku, uporablja le majhno število ljudi.
    Dr. August Dvorak je v svoji raziskavi postavil temeljna načela, ki naj bi jih imela vsaka dobra postavitev tipk. \\

    Namen moje seminarske naloge je preučiti ta načela in z njihovo pomočjo sestaviti optimalno postavitev tipk
    za slovenski jezik.
    Prav tako želim preučiti še druge vire na tem področju, saj je že veliko ljudi sledilo Dvorakovemu zgledu
    in je poskusilo narediti alternativno postavitev za njihov jezik.
    Glede na moje védenje, je moja seminarska naloga prvi poskus prilagajanja postavitve tipk za slovenski jezik.
    Predpostavljam, da bo moja postavitev tipk boljša od sedanje slovenske QWERTZ postavitve tipk. \\

    Pri optimizaciji postavitve bom uporabil besedilo v slovenskem jeziku (korpus),
    ki ga bom s pomočjo programa analiziral, nato pa z njegovo pomočjo izračunal težavnost
    trenutne slovenske postavitve tipk.
    S pomočjo besedila bom postavitev tipk optimiziral in nato izračunal njeno težavnost.
    Obe težavnosti bom primerjal in tako potrdil ali ovrgel svojo hipotezo.

\end{document}