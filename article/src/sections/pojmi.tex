\begin{document}

    \subsection{Postavitev tipk tipkovnice}\label{subsec:postavitev-tipk-tipkovnice}

    Postavitev tipk tipkovnice (ang. \emph{keyboard layout}) se navezuje na fizično postavitev tipk na tipkovnici.
    Pet- ali šestčrkovna imena, kot so na primer QWERTY, QWERTZ in AZERTY, se običajno navezujejo na prvih pet ali šest črk tipkovnice,
    se pravi na prvih pet ali šest črk v prvi zgornji vrstici, ki vsebuje črke, branih od leve proti desni.
    V Sloveniji uporabljamo postavitev QWERTZ, imenovano po zamenjavi črk Y in Z na angleški tipkovnici.
    Prav tako je običajno, da obstaja več različic postavitve tipkovnice v določenem jeziku, ustvarjene pa so bile z zamenjavo nekaterih črk, števk ali znakov.
    Na Linux operacijskih sistemih ima slovenska tipkovnica tri postavitve, angleška pa kar osem,
    se pa lahko število podprtih postavitev spreminja z operacijskim sistemom.

    \subsection{Stanje tipk tipkovnice}\label{subsec:stanje-tipk-tipkovnice}

    Vsaka tipkovnica ima vsaj 3 tako imenovana "stanja" (ang. \emph{keyboard states}).
    Le-ta opisujejo posebne tipke (ang. \emph{modifier keys}), ki so bile pritisnjene ob pritisku navadne tipke.
    Primeri posebnih tipk so \emph{Shift}, \emph{Ctrl}, \emph{Caps Lock} in \emph{Alt}. \\

    \textbf{Prvo stanje} opisuje pritisk navadne tipke brez sočasnega pritiska posebne tipke.

    Primer: pritisk tipke \emph{a} povzroči prikaz črke \emph{a} na zaslonu. \\

    \textbf{Drugo stanje} opisuje pritisk navadne tipke sočasno s pritiskom tipke \emph{Shift}.

    Primer: pritisk tipke \emph{a} in tipke \emph{Shift} sočasno povzroči prikaz črke \emph{A} na zaslonu. \\

    \textbf{Tretje stanje} opisuje pritisk navadne tipke sočasno s pritiskom desne tipke Alt (\emph{AltGr}).

    Primer: pritisk tipke \emph{a} in desne tipke \emph{Alt} sočasno povzroči prikaz črke \emph{æ} na zaslonu. \\

    Zgornji primeri veljajo za Slovensko QWERTZ tipkovnico.

\end{document}