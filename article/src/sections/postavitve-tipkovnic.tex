\begin{document}

  Zaradi različnih potreb uporabnikov obstaja kar nekaj različnih postavitev tipkovnic.
  Najbolj pogosto uporabljene so tiste, ki so v posameznih državah sprejete kot uradne in se uporabljajo v vsakdanjem življenu.
  Alternativnih postavitev tipkovnic je zelo veliko, zato so spodaj omenjene le najpogostejše.
  Skoraj vsaka država uporablja svojo postavitev, večino alternativ pa predstavljajo hobiistične postavitve,
  tj. postavitve, ki jih naredijo zanesenjaki za zabavo ali pa zaradi optimizacije katere druge postavitve.

  Primeri nekaterih hobiističnih tipkovnic so Capewell, Arensito, QWKRFY, QGMLWY, TNWMLC, Norman in Workman.

  \subsection{ABCDE postavitev}\label{subsec:abcde}

  ABCDE postavitev je najbolj osnovna oblika postavitve,
  v kateri so si tipke sledile zaporedno v abecednem vrstnem redu.
  Nekatere prvotne tipkovnice so uporabljale to postavitev, a jo je hitro zamenjala postavitev QWERTY. \\

  \begin{figure}[h]
    \begin{center}
      \includegraphics[width=0.7\linewidth]{resources/layout-abcde.jpg}
      \caption{ABCDE postavitev}
    \end{center}
  \end{figure}

  \newpage
  \subsection{QWERTY postavitev}\label{subsec:qwerty}

  Še danes najpogostejša postavitev, narejena za uporabo na pisalnih strojih.
  Najpogosteje je uporabljena v angleško govorečih državah,
  ki imajo lahko tudi manjše razlike od klasične postavitve QWERTY. \\

  \begin{figure}[h]
    \begin{center}
      \includegraphics[width=0.7\linewidth]{resources/layout-qwerty.png}
      \caption{QWERTY postavitev}
    \end{center}
  \end{figure}

  \subsection{QWERTZ postavitev}\label{subsec:qwertz}

  Manj popularna postavitev, uporabljena večinoma v evropskih državah.
  Ima več oblik, spodaj prikazana pa je tako imenovana južnoslovanska postavitev,
  saj vsebuje črke prilagojene za slovenščino, bosanščino, srbščino in hrvaščino.

  \begin{figure}[h]
    \begin{center}
      \includegraphics[width=0.8\linewidth]{resources/layout-qwertz.png}
      \caption{QWERTZ postavitev}
    \end{center}
  \end{figure}

  \newpage

  \subsection{AZERTY postavitev}\label{subsec:azerty}

  AZERTY postavitev se najpogosteje uporablja v Franciji in Belgiji.
  Podobna je QWERTY postavitvi, a ima nekaj zamenjanih črk in znakov.
  Zanimivo je, da je za izpis števk potrebno držati tipko \emph{Shift}. \\

  \begin{figure}[h]
    \begin{center}
      \includegraphics[width=0.8\linewidth]{resources/layout-azerty.png}
      \caption{AZERTY postavitev}
    \end{center}
  \end{figure}

  \subsection{Dvorak postavitev}\label{subsec:dvorak}

  August Dvorak se je namenil izboljšati QWERTY postavitev, ki je takrat že vladala računalniškemu svetu.
  Po svoji večletni raziskavi je razvil tako imenovano Dvorak postavitev za angleški jezik~\cite{typewriting_behaviour},
  za katero je trdil, da se na njej lažje in hitreje piše,
  saj je zmanjšal razdaljo med tipkami, potrebno za izpis večine angleških besed.

  Kljub temu, da Dvorak postavitev ni uspela zamenjati QWERTY postavitve,
  jo še vedno podpira večina operacijskih sistemov, kot so Linux, macOS, Windows, Android in BSD\@. \\

  \begin{figure}[h]
    \centering
    \includegraphics[width=0.8\linewidth]{resources/layout-dvorak.png}
    \caption{Dvorak postavitev}
  \end{figure}

  \newpage
  \subsection{Programerska Dvorak postavitev}\label{subsec:programerska-dvorak-postavitev}

  Programersko Dvorak postavitev je zasnoval Roland Kaufmann~\cite{programmer_dvorak},
  saj se mu je zdelo, da bi lahko Dz njo programerji hitreje pisali in lažje programirali v angleščini.
  Postavitev je prilagodil večim programskim jezikom,
  kot so Java, C, C\#, Pascal, CSS, XML itd.

  Tipke je prilagodil predvsem po svojem občutku, zaradi česar sta tipki za oklepaj in zaklepaj na levem in desnem kazalcu,
  medtem ko je oglati oklepaj na levem mezincu, oglati zaklepaj pa na desnem prstancu.
  Razlog za takšno postavitev je, da je v nekaterih programskih jezikih za oglatim zaklepajem pogosto znak plus (+),
  ali pa znak minus (-). \\

  \begin{figure}[h]
    \centering
    \includegraphics[width=0.8\linewidth]{resources/layout-programmer-dvorak.png}
    \caption{Programerska Dvorak postavitev}
  \end{figure}

  \subsection{Colemak postavitev}\label{subsec:colemak}

  Colemak je samooklicana modernejša alternativa postavitvama QWERTY in Dvorak,
  zasnovana leta 2006~\cite{colemak}.
  Kljub temu, da je v primerjavi s prej omenjenima postavitvama izboljšana,
  pa ohranja nekaj tipk na istih mestih kot QWERTY, da začetnikom omogoči lažji prehod na postavitev.
  Trenutno je tretja najpopularnejša postavitev za angleške jezike, takoj za QWERTY in Dvorak postavitvama.

  \begin{figure}[h]
    \centering
    \includegraphics[width=0.8\linewidth]{resources/layout-colemak.png}
    \caption{Colemak postavitev}
  \end{figure}

\end{document}