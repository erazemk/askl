\begin{document}

    Po nastanku računalniške tipkovnice, ko so pisalni stroji že zamirali, je dr. August Dvorak ugotovil,
    da QWERTY postavitev ni optimizirana za delo na računalnikih, saj pri njih ni potrebno biti pozoren na zatikanje ročic,
    kot je to običajno pri pisalnih strojih.
    Prsti za pisanje besed potrebujejo na QWERTY tipkovnici opraviti veliko več poti, kot je je potrebno,
    kar zmanjša učinkovitost, poveča potreben čas pisanja besed in lahko dovede do poškodb rok,
    kot je vnetje karpalnega kanala.

    \subsection{Nastanek}\label{subsec:nastanek}

    Dr. August Dvorak je skupaj z dr. Williamom Dealeyem raziskoval črkovne vzorce angleškega jezika in tipkovnico prilagodil tako,
    da se je izognil "napakam", ki jih ima postavitev QWERTY:

    \begin{itemize}
        \item neudobni gibi prstov pri nekaterih besedah,
        \item veliko besed potrebuje preskok prstov iz spodnje čez srednjo vrstico, zaradi postavitve samoglasnikov,
        \item veliko besed je napisanih le z eno roko,
        \item veliko besed je napisanih z levo roko, ki pri večini ljudi ni dominantna,
        \item največji delež tipkanja poteka v zgornji vrstici (okoli 52 \%)~\cite{curse} in ne v srednji \mbox{(okoli 32 \%)}, kar 16 \% tipkanja pa na spodnji, najbolj neudobni vrstici za pritiskanje tipk.
    \end{itemize}

    Postavitev, ki jo je poimenoval po sebi, je dr. August Dvorak ustvaril na podlagi naslednjih dognanj:

    \begin{itemize}
        \item črke naj bodo napisane izmenjaje z obema rokama, kar poveča hitrost in zmanjša pot, ki jo morajo prepotovati prsti,
        \item najpogostejše črke in znaki naj bodo na sredinski vrstici, kar poveča hitrost in učinkovitost pisanja,
        \item najredkeje uporabljene črke naj bodo na spodnji vrstici, ki jo je najtežje doseči,
        \item desna roka naj opravi več tipkanja, saj je večina ljudi desničarjev.
    \end{itemize}

    Dvorak je tipkovnico zasnoval za angleški jezik, a so jo kasneje posamezniki prilagodili tudi za druge jezike,
    kjer so vzorci pisanja drugačni.

    \subsection{Podpora v operacijskih sistemih}\label{subsec:podpora-na-operacijskih-sistemih}

    Čeprav Dvorak postavitev nikoli ni dosegla popularnosti, ki si jo je želel dr. Dvorak,
    jo danes kljub temu podpira večina večjih operacijskih sistemov, kot so Windows, Linux in MacOS,
    na voljo pa je tudi za več jezikov.
    
    \subsection{Prednosti pred ostalimi postavitvami}\label{subsec:prednosti-pred-ostalimi-postavitvami}

    Dvorakova postavitev ima kar nekaj prednosti pred postavitvami kot je QWERTY\@,
    a se je potrebno zavedati, da te prednosti držijo le pri angleščini,
    medtem ko so pri drugih jezikih zanemarljive ali pa jih sploh ni.

    \subsubsection{Preprostost in hitrost učenja}

    Dvorak postavitev se novi uporabnik naučijo veliko lažje kot postavitev QWERTY\@~\cite{dvorak_procon}.
    Učenje je lahko tudi bolj zanimivo, saj uporabniki napišejo veliko besed kar v domači (srednji) vrstici.
    Z dovolj vadbe se lahko uporabniki naučijo uporabljati Dvorak tipkovnico v okoli mesecu dni,
    kar je bilo dokazano z več raziskavami \cite{mwbrooks_opinions}.
    A pokazalo se je tudi, da nekateri novi uporabniki ne pišejo tako hitro, kot so s QWERTY postavitvijo.
    To je pogosto posledica tega, da v opazovanem obdobju niso delali samo z Dvorakovo postavitvijo,
    pač pa tudi s QWERTY\@.

    \subsubsection{Učinkovitost Dvorakove postavitve}

    Učinkost Dvorakove ideje na tem področju je precej vidna.
    Dr. August Dvorak je okoli desetletje preučeval angleški jezik in mu tipkovnico popolnoma prilagodil.
    Ker se večina tipkanja dogaja na srednji vrstici, potrebujejo prsti manj premikanja, kar zmanjša tudi neudobne gibe rok.
    Prav tako postavitev omogoča bolj enakomerno porazdelitev dela med rokama, okoli 56 \% dela opravi desna roka,
    ki je pri večini dominantna, okoli 44 \% pa leva.

    \subsection{Slabosti postavitve tipk}\label{subsec:slabosti-postavitve-tipk}

    \subsubsection{Združljivost}

    Zaradi razširjenosti QWERTY postavitve je večina prenosnih računalnikov in tipkovnic izdelanih s QWERTY postavitvijo,
    kar pomeni, da imajo uporabniki Dvorak postavitve nekaj težav, če uporabljajo več različnih računalnikov.
    Za nekatere uporabnike, ki uporabljajo tekstovna urejevalnike, kot sta \emph{vim} ali \emph{emacs}
    in ki poznajo ukaze po njihovih postavitvah, ne črkah, lahko Dvorak postavitvev oteži delo,
    saj si morajo v programu spremeniti ukaze raznih tipk.
    Nekateri programi prav tako celo ignorirajo sistemske nastavitve in uporabljajo postavitev,
    ki je fizično povezana na tipkovnico, kar pomeni, da bi uporabniki potrebovali fizično Dvorak tipkovnico.

    \subsubsection{Povezanost z angleščino}

    Mogoče največja slabost Dvorak tipkovnice je njena povezanost z angleščino,
    ki zelo zmanjša število uporabnikov in poveča število alternativnih postavitev za druge jezike.

\end{document}