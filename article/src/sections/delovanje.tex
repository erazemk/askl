\begin{document}

  Ne glede na to, da je za nas tipkovnica vsakdanji predmet uporabe, sta njena struktura in delovanje veliko bolj zapletena.
  Vsaka tipkovnica vsebuje svoj procesor, ki zazna kontakt ob pritisku tipke in le tega prevede v določeno kodo (ang. \emph{scancode}).
  Vsaka tipka na tipkovnici je povezana z določeno kodo, ki jo tipkovnica prebere iz tabele kod (ang. \emph{character map}), shranjenih v njenem spominu.
  Kodo nato pošlje v računalnik, ki jo prevede v črke, znake ali ukaze.
  Vsak operacijski sistem ponuja različne načine prevajanja te kode v znake.
  Čeprav tipkovnice v večini uporabljajo enake kode, pa se lahko te vseeno razlikujejo med različnimi izdelovalci.

  \subsection{Linux operacijski sistem}\label{subsec:linux-operacijski-sistem}

  \emph{Linux kernel} oziroma jedro je sestavni del Linux ekosistema.
  Linux, kot ga pozna večina, je dejansko le skupina tako imenovanih distribucij,
  ki temeljijo na Linux jedru.
  Pritisk tipke deluje na naslednjih stopnjah~\cite{arch_wiki}:

  \begin{enumerate}
    \item tipkovnica v računalnik pošlje prej omenjeno kodo (ang. \emph{scancode}), ki računalniku pove, katera fizična tipka je bila pritisnjena,
    \item Linux jedro prevede to kodo o tipki v kodo (ang. \emph{keycode}), ki jo lahko interpretirajo nastavitve tipkovnice - različne jezikovne nastavitve (npr. slovenska, angleška tipkovnica),
    \item nastavljena postavitev prevede kodo v simbol (ang. \emph{keysym}), glede na to, kateri posebni gumbi so bili pritisnjeni (Shift, Caps Lock).
  \end{enumerate}

  \subsection{Windows operacijski sistem}\label{subsec:windows-operacijski-sistem}

  Pri Windows operacijskih sistemih tipkovnice delujejo na naslednji način~\cite{win_keyboard}:

  \begin{enumerate}
    \item tipkovnica pošlje prej omenjeno kodo (ang. \emph{scancode}) v računalnik,
    \item gonilnik za tipkovnice prevede to kodo v drugačno kodo (ang. \emph{virtual-key code}), ki je neodvisna od tipkovnice,
    \item večina tipk se prevede v ASCII ekvivalentne simbole (0 - 9 in a - z), ostale pa v sistemske ukaze (F1 - F12 itd.).
  \end{enumerate}

\end{document}